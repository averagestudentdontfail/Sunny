% Options for packages loaded elsewhere
% Options for packages loaded elsewhere
\PassOptionsToPackage{unicode}{hyperref}
\PassOptionsToPackage{hyphens}{url}
\PassOptionsToPackage{dvipsnames,svgnames,x11names}{xcolor}
%
\documentclass[
  american,
  11pt,
  11pt,
  letterpaper,
  onecolumn]{article}
\usepackage{xcolor}
\usepackage[top=1.2in,bottom=1.2in,left=1.25in,right=1.25in]{geometry}
\usepackage{amsmath,amssymb}
\setcounter{secnumdepth}{5}
\usepackage{iftex}
\ifPDFTeX
  \usepackage[T1]{fontenc}
  \usepackage[utf8]{inputenc}
  \usepackage{textcomp} % provide euro and other symbols
\else % if luatex or xetex
  \usepackage{unicode-math} % this also loads fontspec
  \defaultfontfeatures{Scale=MatchLowercase}
  \defaultfontfeatures[\rmfamily]{Ligatures=TeX,Scale=1}
\fi
\usepackage[]{mathpazo}
\ifPDFTeX\else
  % xetex/luatex font selection
\fi
% Use upquote if available, for straight quotes in verbatim environments
\IfFileExists{upquote.sty}{\usepackage{upquote}}{}
\IfFileExists{microtype.sty}{% use microtype if available
  \usepackage[]{microtype}
  \UseMicrotypeSet[protrusion]{basicmath} % disable protrusion for tt fonts
}{}
\usepackage{setspace}
\makeatletter
\@ifundefined{KOMAClassName}{% if non-KOMA class
  \IfFileExists{parskip.sty}{%
    \usepackage{parskip}
  }{% else
    \setlength{\parindent}{0pt}
    \setlength{\parskip}{6pt plus 2pt minus 1pt}}
}{% if KOMA class
  \KOMAoptions{parskip=half}}
\makeatother
% Make \paragraph and \subparagraph free-standing
\makeatletter
\ifx\paragraph\undefined\else
  \let\oldparagraph\paragraph
  \renewcommand{\paragraph}{
    \@ifstar
      \xxxParagraphStar
      \xxxParagraphNoStar
  }
  \newcommand{\xxxParagraphStar}[1]{\oldparagraph*{#1}\mbox{}}
  \newcommand{\xxxParagraphNoStar}[1]{\oldparagraph{#1}\mbox{}}
\fi
\ifx\subparagraph\undefined\else
  \let\oldsubparagraph\subparagraph
  \renewcommand{\subparagraph}{
    \@ifstar
      \xxxSubParagraphStar
      \xxxSubParagraphNoStar
  }
  \newcommand{\xxxSubParagraphStar}[1]{\oldsubparagraph*{#1}\mbox{}}
  \newcommand{\xxxSubParagraphNoStar}[1]{\oldsubparagraph{#1}\mbox{}}
\fi
\makeatother


\usepackage{longtable,booktabs,array}
\usepackage{calc} % for calculating minipage widths
% Correct order of tables after \paragraph or \subparagraph
\usepackage{etoolbox}
\makeatletter
\patchcmd\longtable{\par}{\if@noskipsec\mbox{}\fi\par}{}{}
\makeatother
% Allow footnotes in longtable head/foot
\IfFileExists{footnotehyper.sty}{\usepackage{footnotehyper}}{\usepackage{footnote}}
\makesavenoteenv{longtable}
\usepackage{graphicx}
\makeatletter
\newsavebox\pandoc@box
\newcommand*\pandocbounded[1]{% scales image to fit in text height/width
  \sbox\pandoc@box{#1}%
  \Gscale@div\@tempa{\textheight}{\dimexpr\ht\pandoc@box+\dp\pandoc@box\relax}%
  \Gscale@div\@tempb{\linewidth}{\wd\pandoc@box}%
  \ifdim\@tempb\p@<\@tempa\p@\let\@tempa\@tempb\fi% select the smaller of both
  \ifdim\@tempa\p@<\p@\scalebox{\@tempa}{\usebox\pandoc@box}%
  \else\usebox{\pandoc@box}%
  \fi%
}
% Set default figure placement to htbp
\def\fps@figure{htbp}
\makeatother



\ifLuaTeX
\usepackage[bidi=basic]{babel}
\else
\usepackage[bidi=default]{babel}
\fi
% get rid of language-specific shorthands (see #6817):
\let\LanguageShortHands\languageshorthands
\def\languageshorthands#1{}
\ifLuaTeX
  \usepackage[english]{selnolig} % disable illegal ligatures
\fi


\setlength{\emergencystretch}{3em} % prevent overfull lines

\providecommand{\tightlist}{%
  \setlength{\itemsep}{0pt}\setlength{\parskip}{0pt}}



 
\usepackage[style=apa,]{biblatex}


\usepackage{booktabs}
\usepackage{longtable}
\usepackage{array}
\usepackage{multirow}
\usepackage{float}
\usepackage{amsmath}
\usepackage{amsfonts}
\usepackage{amssymb}
\usepackage{mathtools}
\usepackage{bm}
\usepackage{microtype}
\usepackage{setspace}
\usepackage{fancyhdr}
\usepackage{titlesec}
\usepackage{caption}
\usepackage{tikz}
\usetikzlibrary{shapes.geometric,arrows,positioning,decorations.pathreplacing}

% Custom colors
\definecolor{sunnyyellow}{RGB}{255, 223, 0}
\definecolor{sunnyblue}{RGB}{30, 144, 255}
\definecolor{sunnygray}{RGB}{248, 249, 250}
\definecolor{clinicalred}{RGB}{220, 53, 69}
\definecolor{clinicalgreen}{RGB}{40, 167, 69}

% Professional section formatting
\titleformat{\section}{\large\bfseries\sffamily\color{NavyBlue}}{\thesection}{1em}{}
\titlespacing*{\section}{0pt}{24pt}{12pt}

\titleformat{\subsection}{\normalsize\bfseries\sffamily\color{NavyBlue}}{\thesubsection}{1em}{}
\titlespacing*{\subsection}{0pt}{18pt}{9pt}

\titleformat{\subsubsection}{\normalsize\bfseries\sffamily\color{NavyBlue}}{\thesubsubsection}{1em}{}
\titlespacing*{\subsubsection}{0pt}{12pt}{6pt}

% Professional header and footer
\pagestyle{fancy}
\fancyhf{}
\rhead{\small\thepage}
\lhead{\small\textit{Case 1a: Peter (Panic Disorder)}}
\renewcommand{\headrulewidth}{0.5pt}
\renewcommand{\headrule}{\hbox to\headwidth{\color{NavyBlue}\leaders\hrule height \headrulewidth\hfill}}
\setlength{\headheight}{14pt}
\makeatletter
\@ifpackageloaded{caption}{}{\usepackage{caption}}
\AtBeginDocument{%
\ifdefined\contentsname
  \renewcommand*\contentsname{Table of contents}
\else
  \newcommand\contentsname{Table of contents}
\fi
\ifdefined\listfigurename
  \renewcommand*\listfigurename{List of Figures}
\else
  \newcommand\listfigurename{List of Figures}
\fi
\ifdefined\listtablename
  \renewcommand*\listtablename{List of Tables}
\else
  \newcommand\listtablename{List of Tables}
\fi
\ifdefined\figurename
  \renewcommand*\figurename{Figure}
\else
  \newcommand\figurename{Figure}
\fi
\ifdefined\tablename
  \renewcommand*\tablename{Table}
\else
  \newcommand\tablename{Table}
\fi
}
\@ifpackageloaded{float}{}{\usepackage{float}}
\floatstyle{ruled}
\@ifundefined{c@chapter}{\newfloat{codelisting}{h}{lop}}{\newfloat{codelisting}{h}{lop}[chapter]}
\floatname{codelisting}{Listing}
\newcommand*\listoflistings{\listof{codelisting}{List of Listings}}
\captionsetup{labelsep=colon}
\makeatother
\makeatletter
\makeatother
\makeatletter
\@ifpackageloaded{caption}{}{\usepackage{caption}}
\@ifpackageloaded{subcaption}{}{\usepackage{subcaption}}
\makeatother
\usepackage{bookmark}
\IfFileExists{xurl.sty}{\usepackage{xurl}}{} % add URL line breaks if available
\urlstyle{same}
\hypersetup{
  pdftitle={Exam Binder: Case 1a - Peter (Panic Disorder)},
  pdfauthor={Clinical Psychology Student},
  pdflang={en-US},
  pdfkeywords={Panic Disorder, Cognitive Behavioural
Therapy, Interoceptive Exposure, Functional Analysis, Case
Formulation, Psychoeducation},
  colorlinks=true,
  linkcolor={NavyBlue},
  filecolor={Maroon},
  citecolor={NavyBlue},
  urlcolor={NavyBlue},
  pdfcreator={LaTeX via pandoc}}


\title{Exam Binder: Case 1a - Peter (Panic Disorder)}
\usepackage{etoolbox}
\makeatletter
\providecommand{\subtitle}[1]{% add subtitle to \maketitle
  \apptocmd{\@title}{\par {\large #1 \par}}{}{}
}
\makeatother
\subtitle{Cognitive Behavioural Therapy Assessment and Intervention
Framework}
\author{Clinical Psychology Student}
\date{2025-06-24}
\begin{document}
\maketitle
\begin{abstract}
This examination binder presents a comprehensive cognitive behavioural
therapy (CBT) framework for assessing and treating panic disorder, using
Peter's case as an exemplar. The document integrates the 5-Stage
Functional Analysis Model with Barlow's Vicious Cycle of Panic theory to
provide a systematic approach to case formulation. Key components
include collaborative assessment techniques, psychoeducation strategies,
interoceptive exposure protocols, cognitive restructuring methods, and
structured homework assignments. The framework emphasizes evidence-based
interventions designed to break the maintenance cycle of panic through
both behavioural and cognitive mechanisms. Essential competencies
covered include assessment skills, psychoeducation delivery, behavioural
technique implementation, cognitive intervention strategies, and
therapeutic homework design.
\end{abstract}


\setstretch{1.5}
\section{Introduction and Overview}\label{introduction-and-overview}

This examination binder presents a comprehensive framework for the
assessment and treatment of panic disorder using cognitive behavioural
therapy (CBT) principles. The case study focuses on Peter, a client
presenting with classic panic disorder symptoms, providing an exemplar
for the systematic application of evidence-based interventions.

The framework integrates five essential competencies: (1) comprehensive
assessment using the 5-Stage Functional Analysis Model, (2)
collaborative psychoeducation based on Barlow's Vicious Cycle of Panic,
(3) behavioural interventions through interoceptive exposure, (4)
cognitive restructuring techniques, and (5) structured homework
assignments for skills consolidation.

\section{CB Assessment: Functional Analysis Using the 5-Stage
Model}\label{cb-assessment-functional-analysis-using-the-5-stage-model}

\subsection{Objective and Theoretical
Framework}\label{objective-and-theoretical-framework}

The primary objective is to conduct a collaborative functional analysis
of Peter's panic using the 5-Stage Model, moving from general
description to specific, shared understanding of the \textbf{Vicious
Cycle of Panic} as it applies to his supermarket incident.

\subsubsection{Barlow's Vicious Cycle of Panic
Model}\label{barlows-vicious-cycle-of-panic-model}

The theoretical foundation rests on Barlow's conceptualization of panic
as a vicious cycle involving catastrophic misinterpretation of bodily
sensations. \textbf{?@fig-panic-cycle} illustrates this model:

\begin{figure}[H]
\centering
\begin{tikzpicture}[
  node distance=2cm and 3cm,
  every node/.style={align=center, text width=2.5cm},
  box/.style={rectangle, draw, thick, fill=blue!10, minimum height=1.2cm},
  cycle/.style={rectangle, draw, thick, fill=red!20, minimum height=1.2cm},
  relief/.style={ellipse, draw, thick, fill=green!10, minimum height=1cm},
  maintain/.style={ellipse, draw, thick, fill=red!30, minimum height=1cm},
  arrow/.style={->, thick, >=stealth}
]

% Main cycle nodes
\node[box] (trigger) {Trigger\\e.g., Stress, Exercise, Caffeine};
\node[cycle, right=of trigger] (threat) {Perceived Threat\\``Oh no, what's this feeling?''};
\node[cycle, below=of threat] (sensations) {Bodily Sensations\\Palpitations, Dizziness};
\node[cycle, left=of sensations] (catastrophic) {Catastrophic Misinterpretation\\``I'm having a heart attack!''};
\node[cycle, above=of catastrophic] (anxiety) {Anxiety/Fear\\SUDS 90/100};

% Safety behaviours and outcomes
\node[box, right=3cm of sensations] (safety) {Safety Behaviours\\\& Avoidance\\e.g., Fleeing store,\\Avoiding gym};
\node[relief, below=of safety] (shortterm) {Short-Term\\Relief};
\node[maintain, above=of safety] (longterm) {Long-Term:\\Fear Increases};

% Arrows for main cycle
\draw[arrow] (trigger) -- (threat);
\draw[arrow] (threat) -- (sensations);
\draw[arrow] (sensations) -- (catastrophic);
\draw[arrow] (catastrophic) -- (anxiety);
\draw[arrow] (anxiety) -- (threat);
\draw[arrow, bend left=20] (sensations) to (anxiety);

% Safety behaviour arrows
\draw[arrow] (anxiety) -- (safety);
\draw[arrow] (safety) -- (shortterm);
\draw[arrow] (safety) -- (longterm);
\draw[arrow, dashed, bend right=30] (shortterm) to (longterm);

% Vicious cycle highlight
\draw[thick, red, dashed, rounded corners] 
  ([xshift=-0.3cm,yshift=0.3cm]threat.north west) rectangle 
  ([xshift=0.3cm,yshift=-0.3cm]anxiety.south east);

\node[above=0.2cm of threat, text=red, font=\textbf] {Vicious Cycle};

\end{tikzpicture}
\caption{Barlow's Vicious Cycle of Panic: Theoretical Model}
\label{fig-panic-cycle}
\end{figure}

This diagram demonstrates how an initial trigger leads to bodily
sensations, which are catastrophically misinterpreted. This
misinterpretation spikes fear, amplifying bodily sensations and creating
a vicious feedback loop. Safety behaviours provide short-term relief but
prevent new learning and maintain the cycle long-term.

\subsection{Stage-by-Stage Assessment
Procedure}\label{stage-by-stage-assessment-procedure}

\subsubsection{Stage 1: Identify Problem Behaviours and Context
(\textasciitilde2
minutes)}\label{stage-1-identify-problem-behaviours-and-context-2-minutes}

\textbf{Action:} Define the problem in concrete, behavioural terms. What
does the client do/not do? What are the specific physiological symptoms?

\textbf{Assessment Script:} \textgreater{} ``Peter, thank you for coming
in. My goal today is to build a really clear picture of these panic
attacks with you. You've mentioned a few key things that are happening:
you're experiencing intense physical feelings like a \textbf{pounding
heart, dizziness, and tremors}. And in terms of what you do, it sounds
like you've started \textbf{avoiding the gym}, and you mentioned you
\textbf{left your trolley and fled the supermarket}. Have I got that
right? Are there any other key behaviours we should put on our map?''

\subsubsection{Stage 2: Clarify Antecedents (Triggers and Modifiers)
(\textasciitilde3
minutes)}\label{stage-2-clarify-antecedents-triggers-and-modifiers-3-minutes}

\textbf{Action:} Investigate specific internal and external events
preceding the problem, using the supermarket incident as anchor point.

\textbf{Assessment Script:} \textgreater{} ``Okay, let's go back to the
supermarket. Can you walk me through the moments \emph{just before} your
heart started pounding? We call these `triggers.' Were you feeling
rushed? Stressed? What was the first sign that things were about to
shift?''

\textbf{Probing for Modifiers using BASIC-P framework:}

\begin{itemize}
\tightlist
\item
  \textbf{Affect:} ``You mentioned work has been very stressful with the
  new boss. How does your general stress level on a given day seem to
  affect the likelihood of an attack?''
\item
  \textbf{Physiological:} ``It seems like any activity that increases
  your heart rate, even normal exercise, can act as a trigger now. Is
  that right?''
\end{itemize}

\subsubsection{Stage 3: Clarify Consequences (The Maintenance Cycle)
(\textasciitilde3
minutes)}\label{stage-3-clarify-consequences-the-maintenance-cycle-3-minutes}

\textbf{Action:} Uncover the Vicious Cycle by linking trigger to
thought, emotion, and subsequent behaviours that maintain the problem.

\textbf{Eliciting the Hot Thought:} \textgreater{} ``Okay, so your heart
starts to pound. In that exact instant, what is the \emph{meaning} you
give to that sensation? What's the most frightening thought that goes
through your mind?''

\emph{Target: Catastrophic Misinterpretation (e.g., ``I'm going to
collapse,'' ``I'm having a heart attack'')}

\textbf{Linking Thought to Emotion/Physiology:} \textgreater{} ``And
when that thought `I'm having a heart attack' hits, what happens to your
fear on that 0-100 scale? \ldots{} And what about the pounding in your
chest? Does it get worse, stay the same, or get better after that
thought?''

\textbf{Behavioural Consequence and Negative Reinforcement:}
\textgreater{} ``So in response to all that, you left the store. Tell
me, what happened to your anxiety level once you were outside? \ldots{}
It probably went down, which is a huge relief. Here's the tricky part of
panic: because leaving made you feel better, what did that teach your
brain about the supermarket? Did it teach it that the supermarket is a
safe place, or a dangerous place you were right to escape from?''

\subsubsection{Stage 4: Identify Strengths, Resources and Coping
(\textasciitilde1
minute)}\label{stage-4-identify-strengths-resources-and-coping-1-minute}

\textbf{Action:} Actively search for positive factors to create balanced
view and instill hope.

\textbf{Assessment Script:} \textgreater{} ``Peter, while this cycle
sounds exhausting, I also want to acknowledge what's working \emph{for}
you. You have a very supportive wife who is a nurse. You've successfully
been through CBT for this before, which is a huge strength---it means
you are capable of getting better. And despite all this, you're still
going to work and you came here today to get help. That shows real
determination.''

\subsubsection{Stage 5: Integrate into Preliminary Formulation
(\textasciitilde1
minute)}\label{stage-5-integrate-into-preliminary-formulation-1-minute}

\textbf{Action:} Verbally summarise the Vicious Cycle model using
gathered information to make it specific to Peter.

\textbf{Collaborative Summary:} \textgreater{} ``So, if we put all the
pieces together, we get a clear map. It seems a \textbf{Trigger}, like
background stress, leads to a normal \textbf{Bodily Sensation}. Your
brain then has a \textbf{Catastrophic Thought}, `I'm having a heart
attack.' This thought understandably creates intense \textbf{Fear},
which floods your body with adrenaline and makes the sensations even
worse. To save yourself, you engage in a \textbf{Safety Behaviour} like
fleeing the store. This gives you immediate relief, but accidentally
reinforces the initial belief that the situation was truly dangerous,
making the whole cycle more likely to happen again. This is a classic,
well-understood pattern, and the most important thing to know is that we
have very effective, evidence-based tools to break you out of this trap.
Does this picture feel right to you?''

\section{Psychoeducation: The Vicious Cycle
Model}\label{psychoeducation-the-vicious-cycle-model}

\subsection{Objective and Key
Principles}\label{objective-and-key-principles}

\textbf{Objective:} Explain the `Vicious Cycle of Panic' model and
treatment rationale in a clear, collaborative, and non-pathologizing
manner using analogy and visual aids.

\textbf{Key Principles:} Proceed from known to unknown; use analogies;
link to treatment rationale.

\subsection{Psychoeducation Procedure}\label{psychoeducation-procedure}

\subsubsection{Step 1: Introduce the Model and
Analogy}\label{step-1-introduce-the-model-and-analogy}

\begin{quote}
``Peter, based on the map we just built, I'd like to share a model that
explains this panic trap. Think of your body's anxiety system like a
highly sensitive smoke alarm. A good smoke alarm is meant to go off when
there's a real fire. But sometimes, an alarm can become faulty and go
off just from a bit of steam from the shower. It's giving a real alarm,
but to a false danger. That's exactly what's happening with panic.''
\end{quote}

\subsubsection{Step 2: Draw and Explain the
Cycle}\label{step-2-draw-and-explain-the-cycle}

\textbf{The Trigger (The `Steam'):} \textgreater{} ``It starts with what
we call a trigger - a normal bodily sensation like a fast heartbeat from
climbing stairs or feeling stressed.''

\textbf{The Misinterpretation (The `Faulty Sensor'):} \textgreater{}
``But your `alarm system', because of that scary first attack, now
misinterprets that `steam' as a `fire'. It tells you, `This is
dangerous, it's a heart attack'.''

\textbf{The Adrenaline Rush (The `Loud Alarm'):} \textgreater{} ``When
your brain thinks there's a fire, it floods your body with adrenaline to
prepare you to fight or flee. This is why your heart pounds even harder,
you feel dizzy, and you feel that overwhelming urge to escape. It's a
real physical response, but to a false alarm.''

\textbf{The Reinforcement (Confirming the Faulty Belief):}
\textgreater{} ``And here's the trap. Because you escape the situation
(like leaving the gym), your anxiety goes down. Your brain then thinks,
`Whew, I got out just in time. Escaping is what saved me.' So it
reinforces the belief that the situation was truly dangerous, making the
alarm even more sensitive for next time.''

\subsubsection{Step 3: Link Directly to Treatment
Plan}\label{step-3-link-directly-to-treatment-plan}

\textbf{Rationale for Interoceptive Exposure (IE):} \textgreater{} ``So,
to fix this, we need to recalibrate your alarm system. We're going to do
exercises that deliberately create the `steam'---the physical
sensations---right here in this safe room. This will allow your brain to
experience the sensations over and over without any `fire' happening,
until it learns they are not dangerous.''

\textbf{Rationale for Cognitive Restructuring (CR):} \textgreater{} ``At
the same time, we're going to work on the `faulty sensor' itself. We
will examine those scary thoughts, look at the actual evidence for and
against them, and develop more realistic ways of thinking about the
sensations.''

\subsubsection{Step 4: Check Understanding and Instill
Hope}\label{step-4-check-understanding-and-instill-hope}

\begin{quote}
``This two-pronged approach is the most effective, evidence-based way to
treat panic. It puts you back in the driver's seat. How does this
explanation fit with your experience?''
\end{quote}

\section{Behavioural Techniques: Interoceptive
Exposure}\label{behavioural-techniques-interoceptive-exposure}

\subsection{Primary Technique and Core
Rationale}\label{primary-technique-and-core-rationale}

\textbf{Technique:} \textbf{Interoceptive Exposure (IE)}

\textbf{Core Rationale:} Break the conditioned fear response to internal
bodily sensations through \textbf{habituation} and \textbf{extinction
learning}. We directly target the catastrophic misinterpretation of
physical symptoms by proving through direct experience that they are
harmless.

\subsection{Detailed Procedure}\label{detailed-procedure}

\subsubsection{1. Psychoeducation and
Rationale}\label{psychoeducation-and-rationale}

Ensure Peter understands the rationale: we are deliberately triggering
false alarms to recalibrate them.

\subsubsection{2. Hierarchy Construction}\label{hierarchy-construction}

Collaboratively build a graduated hierarchy of exercises that elicit his
specific feared sensations---palpitations and breathlessness. Order from
least to most anxiety-provoking, rated on 0-100 SUDS scale. This
empowers him and ensures we start at manageable level (e.g., 40/100
SUDS).

\subsubsection{3. In-Session Exposure}\label{in-session-exposure}

Begin with first item in-session, for example, `marching on the spot for
60 seconds'. Instruct him to focus on sensations without trying to
suppress or distract from them.

\subsubsection{4. Monitoring}\label{monitoring}

During exercise, monitor his SUDS. Goal is to stay with sensation until
anxiety peaks and begins to naturally decrease by about 50\%,
demonstrating habituation.

\subsubsection{5. Blocking Safety
Behaviours}\label{blocking-safety-behaviours}

Critically, instruct him to drop any safety behaviours such as checking
pulse, deep breathing, or telling himself `it's okay'. These behaviours
interfere with extinction learning process.

\subsubsection{6. Debrief and
Reappraisal}\label{debrief-and-reappraisal}

After each exposure, debrief. Ask: ``What did you notice? Your heart
pounded, but what was the catastrophic outcome? Did you collapse?'' This
consolidates new, non-catastrophic learning.

\subsection{Sample Hierarchy for
Peter}\label{sample-hierarchy-for-peter}

Table~\ref{tbl-exposure-hierarchy} presents a graduated hierarchy of
interoceptive exposure exercises tailored to Peter's specific feared
sensations:

\begin{longtable}[]{@{}
  >{\raggedright\arraybackslash}p{(\linewidth - 6\tabcolsep) * \real{0.1034}}
  >{\raggedright\arraybackslash}p{(\linewidth - 6\tabcolsep) * \real{0.1724}}
  >{\raggedright\arraybackslash}p{(\linewidth - 6\tabcolsep) * \real{0.3103}}
  >{\centering\arraybackslash}p{(\linewidth - 6\tabcolsep) * \real{0.4138}}@{}}
\caption{Sample Interoceptive Exposure
Hierarchy}\label{tbl-exposure-hierarchy}\tabularnewline
\toprule\noalign{}
\begin{minipage}[b]{\linewidth}\raggedright
Step
\end{minipage} & \begin{minipage}[b]{\linewidth}\raggedright
Exercise
\end{minipage} & \begin{minipage}[b]{\linewidth}\raggedright
Feared Sensation
\end{minipage} & \begin{minipage}[b]{\linewidth}\centering
Predicted SUDS (0-100)
\end{minipage} \\
\midrule\noalign{}
\endfirsthead
\toprule\noalign{}
\begin{minipage}[b]{\linewidth}\raggedright
Step
\end{minipage} & \begin{minipage}[b]{\linewidth}\raggedright
Exercise
\end{minipage} & \begin{minipage}[b]{\linewidth}\raggedright
Feared Sensation
\end{minipage} & \begin{minipage}[b]{\linewidth}\centering
Predicted SUDS (0-100)
\end{minipage} \\
\midrule\noalign{}
\endhead
\bottomrule\noalign{}
\endlastfoot
1 & March on the spot for 1 minute & Palpitations & 40 \\
2 & Tense all muscles for 1 minute & Body tension, tingling & 50 \\
3 & Run up and down stairs for 1 flight & Breathlessness, palpitations &
60 \\
4 & Spin in a chair for 1 minute & Dizziness, unreality & 70 \\
5 & Hyperventilate for 60 seconds & Breathlessness, dizziness, tingling
& 80 \\
\end{longtable}

\subsection{Critical Considerations}\label{critical-considerations}

\textbf{Medication Considerations:} Address his use of anxiolytic
medication. Explain that taking a benzodiazepine before exposure would
chemically suppress the anxiety we need to work with, making the
exercise ineffective. Agreement needed that he would not use it as
safety behaviour before sessions or homework.

\section{Cognitive Techniques: Socratic Dialogue and Behavioural
Experiments}\label{cognitive-techniques-socratic-dialogue-and-behavioural-experiments}

\subsection{Primary Technique: Socratic
Dialogue}\label{primary-technique-socratic-dialogue}

\subsubsection{Target Belief for
Intervention}\label{target-belief-for-intervention}

\textbf{Belief:} ``When my heart starts pounding, it means I'm having a
heart attack and I might die.'' \textbf{Conviction:} 95\% during panic

\subsubsection{Step-by-Step Procedure}\label{step-by-step-procedure}

\paragraph{1. Isolate the Thought and Set the
Frame}\label{isolate-the-thought-and-set-the-frame}

\begin{quote}
``Peter, let's focus on that specific thought we identified earlier:
`When my heart pounds, I'm having a heart attack.' I'm not here to tell
you you're wrong, but I'm curious to explore it with you. Can we put
that thought under a microscope for a few minutes?''
\end{quote}

\paragraph{2. Gather Evidence For}\label{gather-evidence-for}

\begin{quote}
``What's the evidence that supports this thought? What makes it feel so
true in the moment?''
\end{quote}

\emph{Expected responses: intensity of feelings, ``it feels just like
what I've read about,'' ``I had to go to the ED''}

\paragraph{3. Gather Evidence Against}\label{gather-evidence-against}

\begin{quote}
``That's compelling evidence. Now let's play devil's advocate for a
moment. What's the evidence \emph{against} this thought? What did the
doctors at the Emergency Department tell you after they ran the tests?
How many times have you had this sensation and this thought? And how
many times has it actually been a heart attack?''
\end{quote}

\paragraph{4. Introduce Alternative
Explanations}\label{introduce-alternative-explanations}

\begin{quote}
``Okay, so it has happened hundreds of times and has never been a heart
attack. Let's think about other, non-dangerous reasons a person's heart
might pound?''
\end{quote}

\emph{Guide towards: exercise, stress, caffeine, excitement, anxiety
itself}

\paragraph{5. Use the Double Standard
Technique}\label{use-the-double-standard-technique}

\begin{quote}
``This is a really helpful question. You mentioned your wife Uma is a
nurse. If she came home from a stressful shift and said `My heart is
racing,' what would be your first thought about what was happening to
her? Would you assume it was a heart attack?''
\end{quote}

\paragraph{6. Develop a More Balanced
Thought}\label{develop-a-more-balanced-thought}

\begin{quote}
``So, given that the medical tests were all clear, that it's happened
hundreds of times without being a heart attack, and that there are many
non-dangerous explanations for a pounding heart, what could be a more
balanced or helpful thought to have in that moment?''
\end{quote}

\emph{Guide towards: ``My heart is pounding. This is just my body's
response to anxiety. It's uncomfortable but not dangerous. It will
pass.''}

\paragraph{7. Test the New Thought}\label{test-the-new-thought}

\begin{quote}
``How does that new thought feel? Let's rate its believability from
0-100\%.''
\end{quote}

\subsection{Alternative Cognitive Technique: Behavioural
Experiment}\label{alternative-cognitive-technique-behavioural-experiment}

\subsubsection{Rationale}\label{rationale}

If Socratic dialogue isn't fully effective, a more powerful cognitive
technique would be a \textbf{Behavioural Experiment}. Its purpose is to
experientially test the validity of his prediction in real-world
setting, providing `gut-level' evidence more powerful than verbal
discussion alone.

\subsubsection{Procedure}\label{procedure}

\textbf{Target Belief:} ``If I go to the gym and do my old workout, I
\emph{will} have a full-blown panic attack and have to be taken to
hospital.'' (Conviction: 80\%)

\textbf{Experiment Design:} - \textbf{Task:} Go to gym together -
\textbf{Specific Step:} 10 minutes on treadmill at moderate pace -
\textbf{Operational Definition:} Define what ``full-blown panic attack''
means for measurement

\textbf{Learning Outcome:} Gather data to see if prediction comes true.
Review outcome and learning (e.g., ``I felt anxious and my heart
pounded, but I did not have a full-blown attack or collapse''), then
re-rate conviction in original belief.

\section{Learning Discovery and Homework
Assignment}\label{learning-discovery-and-homework-assignment}

\subsection{Homework Tasks}\label{homework-tasks}

\subsubsection{Behavioural Task: Interoceptive Exposure
Practice}\label{behavioural-task-interoceptive-exposure-practice}

\textbf{Assignment:} Practice the first interoceptive exposure exercise
completed in session: \textbf{marching on the spot for one full minute,
three times every day.}

\textbf{Monitoring:} Use \textbf{Situational Exposure Diary} to track: -
SUDS before, during, and after - Observations and discoveries - Creates
clear, structured task

\subsubsection{Cognitive Task: Thought Change
Record}\label{cognitive-task-thought-change-record}

\textbf{Assignment:} Begin using a \textbf{Thought Change Record}.

\textbf{Objective:} Be a detective---whenever noticing anxiety, simply
`catch' the thought that came just before and write it down.

\textbf{Note:} Not challenging thoughts yet, just building awareness.

\subsection{Rationale for Homework}\label{rationale-for-homework}

\subsubsection{Interoceptive Exposure
Rationale}\label{interoceptive-exposure-rationale}

The repeated \textbf{Interoceptive Exposure} practice is like physical
therapy for his brain's alarm system. Each repetition without
catastrophe weakens the old fear-pathway and strengthens a new
safety-pathway. It is the most direct way to achieve habituation and
reduce fear of sensations.

\subsubsection{Thought Record Rationale}\label{thought-record-rationale}

The \textbf{Thought Record} serves a different purpose. It helps him
practice the skill of metacognition---seeing thoughts as mental events
rather than objective facts. This is the foundational skill required
before we can effectively evaluate and restructure thoughts in later
sessions.

\textbf{Ultimate Goal:} The homework empowers him to become his own
therapist, as real change happens in the 167 hours between sessions, not
just the one hour together.

\subsection{Collaborative
Problem-Solving}\label{collaborative-problem-solving}

\subsubsection{Anticipating Barriers}\label{anticipating-barriers}

\begin{quote}
``Peter, on a scale of 0 to 100, how confident are you that you can
complete this homework? What might get in the way?''
\end{quote}

\subsubsection{Plan B Development}\label{plan-b-development}

If low confidence expressed, create \textbf{Plan B}: \textgreater{} ``If
marching for a full minute feels like too much, what's a step back that
still feels like a step forward? Perhaps 30 seconds?''

\textbf{Key Principle:} ``There is no such thing as unsuccessful
homework, because we learn something every time.'' This encourages
effort over perfect performance and builds self-efficacy.

\section{Conclusion}\label{conclusion}

This comprehensive framework demonstrates the systematic application of
CBT principles to panic disorder treatment. The integration of
functional analysis, psychoeducation, behavioural interventions,
cognitive techniques, and structured homework provides a evidence-based
approach to breaking the vicious cycle of panic. The collaborative
nature of all interventions ensures client engagement while building
skills for long-term recovery and relapse prevention.

The case exemplifies how understanding the maintenance mechanisms of
panic disorder enables targeted interventions that address both the
cognitive and behavioural components of the disorder, ultimately
empowering clients to become their own therapists.


\printbibliography



\end{document}
